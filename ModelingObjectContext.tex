%------------------------------------------------------------------------
\chapter{Modeling Object Context}
\label{ModelingObjectContext.ch}
%------------------------------------------------------------------------
\section{Problem Statement}
\label{ProblemStatement.sec}
% here we need to define context of objects
%learn the model
As discussed in the previous Chapter, in order to facilitate object detection and provide a means to classify places based on key objects 
for a place category, in this thesis we proposed a structure and a method to build models for object context. 
%In this thesis we present an idea and a model to represent 3D context of object. As discussed in previous chapter, This model is 
%a very useful tool for both object detection and place categorization in which object information is used.

Based on the definition stated in Chapter \ref{Introduction.ch}, by context of the object we are referring to the surfaces 
in a neighborhood around it(Figure \ref{contextExample.figure}). 
The problem that is addressed in this research, is how to define a suitable feature descriptor to represent those surfaces 
as the context of the object class. 
The trained model would be applied on pointclouds of new scenes to determine locations in the scene that are likely
to contain the object (regardless of an object being actually present in thoes locations).
The expected result, is a Likelihood map for different regions of a scene (pointcloud) that corresponds to the presense of an object class.





\section{Motivation and Applications}
\label{MotivationandApplications.sec}
A motive for this work is to achieve a complete and robust 3D descriptor for places based on object contexts 
rather than objects. 
When a human agent enters a room, intuitively he is able to decide about the category of the place based on a simple and 
shallow perception, with a good confidence. 
Even if the room is not furnished yet and there is not many objects in it to help the agent in determining its 
category. 
Perhaps more than presence of some objects in a place, its the possibility of their presence that helps us 
to determine the category of the place. Geometrical and spacial information about different regions in a place can help us to imagine what object categories can be placed in it. In an other word, what sort of object context can be found in that place. 
For sure other information adds up to this knowledge to make a human able to distinguish between places, such as general 
shape and size of the place also its location regarding other places. 
A model for an object context can provide us with the likelihood of object presence.
%Although, the Context Model, which we are talking about in this work, is not enough for making this decision, but is a useful 
%prior.

Among the applications that is considerable for the model, most obvious ones are as follows:


\subsubsection*{Object Detection}

Object detection is an expensive and complex task. Searching for context of 
an object is, most of the times, significantly easier than the object itself. Specially, when the searched object is in a rather small scale comparing to the search area. 
When the context is detected, then the search for the actual object is done in a reduced area, where is predicted as the context. 
In addition, false positives are avoided or at least reduced. For example, when a table is detected as the context for a telephone, any similar shape which might be on the floor will not be considered by the object detector, which otherwise would have detected it as a candidate. 

\subsubsection*{Place Classification}
 
      A very important piece of information that can help a robotic system to be able to distinguish between different place categories 
is knowing about the objects which are expected to be found in a specific category of a place. 

If the robot is looking for a kitchen, it helps if it knows about some objects that are most probable to be found in it and 
perhaps not in other places. 
For instance a kitchen sink is such a discriminative object. 
If the robot can find such an object in a place, it can obtain a good confidence about that place to be a kitchen. 

{\bf\it Viswanathan et al.} in \cite{P.Viswanathan}, also  pointed to the fact that object detection is a good prior for place 
categorization. {\bf\it T. Southey and J. J. Little} in \cite{southey2006object} argued that, {\it rooms} are defined by their geometric 
properties, while definition of {\it places} is based on objects they contain and activities takes place in them.  

On the other hand, the place classification system would be more robust if it is able to recognize the place, using object information, 
even when the actual object is not present in the place, but the presence is expected.
Here is where the contextual model of object finds it's importance. When the context is present, there is no need to detect the actual 
object for the purpose of classifying the place.

 
 
 \subsubsection*{Object Placement} 
 
 An intuitive application for the model of object context is finding a suitable location or surface to put an object.
 When a robot knows the context model of an object it can place it correctly by finding matches of that model in its 
 surrounding.
 For instance, a robot that performs typical household chores, like cleaning or serving guests can benefit from it.  
 
% From here to end of the chapter is moved to chapter 3 (which become 2) on April 21, 2021
 \section{Challenges}
\label{Challenges.sec}

% \subsection{Challenges Related to Applications }%(Where context can help)}
% \label{ChallengesRelatedtoApplications.ssec}
% 
% \subsubsection{Place Classification}
% \label{placeclassification.ssec}
% when we have a context model of key objects for a place class we would have a much simpler job to decide about class of 
% the place based on those context rather than objects themselves. If in the search time the key objects are not present 
% in the scene most probability the contexts would be present. 
% Finding object classes which are discriminative for places is not easy for all place categories. In addition, context model
% for different object classes can be similar which decreases certainty of place classification.  
% 
% \subsubsection{Object Detection}
% \label{objectdetection.ssec}
% Object detection can be challenging when the size of the object compared to the size of the scene is very small. 
% Illumination conditions, occlusion and pose of the object can affect the time needed for the search and performance of the system. 
% 
% Identifying potential locations for the object within the scene can drastically reduce the search space. 
% Detecting a table as a context for cups is easier due to its scale and is less affected by illumination and 
% occlusion. 
% When the table is detected, recognizing objects on the table is facilitated regardless of the challenges mentioned above.

% \subsubsection{Where to put the Object}
% \label{wheretoputtheobject.ssec}
% % In tasks that involve robot to put an item it is carrying to some location , it is needed to provide information to the 
% % robot about where is suitable place to put that type of object, context model is a very good tool in this scenario.
% If the context model provided for an object class, which a robot wants to find a place to put it, is not general enough
% it can cause a problem. For instance, in a scenario that a robot is looking for any suitable surface, on which it is possible 
% to put a cup, if the context model is just suggesting a table as a suitable place, it confuses the robot. 
% There may be no table around, but flat surfaces which can hold a cup.

%\subsection{Challenges Related to Modeling the Object Context}

There are challenges regarding modeling the object context that follows:

\begin{itemize}
%  \item The ability of the features to separate context from non-context points. The features that are used to train the context model
%    should be discriminative to some extent. The context, as is described in Chapter \ref{Introduction.ch}, usually is not a single
%    body. It consists of separate surfaces or even objects, which reduces the distinctiveness of the features.  
%    
 \item Train set quality: Employing a train set without enough generalization. There could be situations and scenes which are completely different from
	what have been considered in train set to build the context model.
	
	\item Annotation of objects in Training data: An easy and reliable tool was needed to annotate 3D objects in pointclouds.
	
 \item Pointcloud quality: The quality of pointclouds is a critical issue. The amount of smoothness in the pointcloud and missing points have a 
 significant effect on the extracted features, which are used to describe surfaces, and as a result, on the context model. 
 When pointclouds are being captured, if due to the viewpoints of the sensor some parts of a surface are occluded, 
 the resulting pointcloud might lack important points.
 The geometry that is extracted from such surfaces is not accurate and in some cases are wrong.
 
 \item Significant difference in the number of positive samples (feature vectors extracted from a region with annotated object presence) against negative ones, in feature vectors extracted form training pointclouds.
 
\end{itemize}

% First Challenge was the feature to use in building the model, looked into different feature types and also different combinations 
% of features to get a suitable construction for our purpose. 
% Another issue was deciding about what to be considered as the context for the object, 
% we needed to reach a fine structure for from where and how to extract features for each object.
% 
% After deciding about the structure of the feature vector it was also needed to do some pre-process on the raw data we had 
% before extracting the features, the details would be described in chapter \ref{3DModelofObjectContext.ch}.
% 
% for training the model we needed annotated data, in the platform I was using there was no suitable annotation tool so 
% I needed to develop one. Then choosing suitable kernels for SVM as the learning method I chose, also parameters each 
% of which had a very significant effect on the results.
% 
% a big issue here was the fact that, we were receiving a lot of non context samples (we will call them as negative samples) 
% from the annotations and a few context samples (Positive). There could be samples among negative dataset which are quite 
% similar to positive ones based on their feature values which are labeled as negative just because were not extracted from 
% the annotated region. This can cause confusion for the classifier. More details would be discussed in Chapter \ref{3DModelofObjectContext.ch}

\subsection{2D vs 3D context, benefits of 3D}
Popularity of devices such as Microsoft Kinect, has made 3D data widely accessible. 
Therefore, many researchers and academics have shown great desire to use 3D information in their researches, 
which gives a lot more information about our surrounding compared to 2D data. 
Using 3D data has made it simple to capture geometrical information in addition to visual properties. 
From the output of this type of sensors, we can easily and directly have access to depth information and 3D 
coordinates of any points that are perceived.

Geometrical features, which are obtained from 3D information, enhances performance of systems significantly in tasks like feature 
based object detection compared to relying only on visual information. 
Particularly, in the problem we are dealing with, to build a model for object context, depending on features which only encode 
changes in intensity and color is not enough. The geometry of surfaces needed to be analyzed and 3D information facilitates and improves this analysis.
