%------------------------------------------------------------------------
\chapter{Conclusion and future work}
\label{Conclusion.ch}
%------------------------------------------------------------------------
In this thesis, we proposed an idea and a method to build 3D model of object context using geometrical features extracted
from point-clouds of a scene or room.
Point-clouds are produced using data from kinect sensor with its limitations.
A set of point-clouds is gathered from different place categories including different objects to be used for experiments.

A metric for evaluation of the results is proposed in a way that it can assess the results as they are expected to be.
The context prediction of the system are expected to be a $superset$ of positive samples in train set.
In spite of the few number of point-clouds and instances of the objects used in experiments the results are quite acceptable. 

In addition a method for employing the object context model in place categorization is proposed and analytically discussed.
This work is unique regarding the suggestion of using 3D context model of objects in building a descriptor for places and also the use of full 
point-clouds of a place. 
These full point-clouds can be extended to a 3D map of an environment.

\section{Future work}

There are some suggestions for improvements and some potential future works which are mentioned here.
%\subsection*{improvements}
\begin{itemize}
 \item The quality of the point-clouds are an important factor, so using better means and methods to capture and build 
 point-clouds with higher quality makes an improvement in results. Doing some pre-process like smoothing and removing noise
 is even a more practical improvement.
 \item Larger train set with more variety of places and object instances is essential. Gathering data in a smart way
  for the train set which considers several possible situations can be more efficient than just increasing the number of samples.
 \item Some more features can be added to make the model more descriptive and discriminative. 
  Visual features should be considered for this improvement.
 \item A comparative evaluation can be done between object detection by a known high performance detector with using context model
  and without.
 \item Develop the system to be able to run in real time and on-line situations.
 \item Employ and experiment place categorization using this model and the proposed method.
\end{itemize} 
