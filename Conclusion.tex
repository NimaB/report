%------------------------------------------------------------------------
\chapter{Conclusion and future work}
\label{Conclusion.ch}
%------------------------------------------------------------------------
In this thesis, we proposed a method to build 3D models of objects' context using geometrical features extracted
from pointclouds of a scene or room.
Point-clouds are produced using data from the Kinect sensor.
A set of pointclouds is gathered from different place categories, including different objects to be used for experiments.

Also, a metric and an evaluation method are proposed to assess the results as they are expected to be.
% The context prediction of the system is expected to be a $superset$ of positive samples in the train set.
% Despite the few pointclouds and instances of the objects used in experiments, the results are acceptable.  

In addition, a method for employing the object context model in place categorization is proposed and analytically discussed.
This work is unique regarding the suggestion of using 3D context models of objects in building a descriptor for places and also the use of full 
pointclouds of a place. 
These full pointclouds can be extended to a 3D map of an environment.

\section{3D Object Context for place Classification}
\label{3D Object Context for place Classification.sec}

In this section, we discuss the proposed idea of using object context in classification of places.
Place classification in here, mainly mean to determine the category of a visited place so that some resulting semantics 
can be added to and understood by the system.

Adding semantic information to maps, to be used by mobile robots, enhances human-robot interaction.
It helps in mobile robots localization and object detection.
The ability of a robot to understand semantics of space and associate spatial locations with semantic terms 
such as $kitchen$ or $corridor$, provides a more clear idea for the robot and its users of its location than a pure 
topological or metric position \cite{pronobis2011phd}.   

Regarding object detection, having semantic information of a place in association with a ground truth about the objects expected
to be present in that place, makes the search for an object faster and more successful.
It helps robotic systems to communicate with humans and interpret environments built by and for them.

Concepts such as $office$ or $kitchen$ are different categories for a room based on its functionality.
Other categorizations can be  based on its spatial properties such as its shape, size or general appearance.

Figure ~\ref{semanticmapping.figure} shows an overview of a semantic mapping system proposed by Pronobis and Jensfelt in \cite{pronobis2011phd}.
As it is illustrated in this picture, inputs to this system are models of objects, shape, size and appearance in the company of a 
common-sense knowledge database.
It is considered as a future work for this thesis that 3D $context$ $models$ replace $object$ $models$ in this system and with some improvements 
a more robust and complete representation for places would be achieved.

\begin{figure}[t]
  \centerpsw{semanticmapping}{0.85\columnwidth}
  \caption[Overview of a semantic mapping system]
  {Overview of a semantic mapping system by Pronobis in \cite{pronobis2011phd}(best viewed in color).}
  \label{semanticmapping.figure}
\end{figure}

\subsection{Brief overview of approaches to place classification}
\label{OverviewPlaceClassification.sec}
 Place categorization based on vision in its first stages was focused on classifying single 2D images of an indoor or 
 outdoor scene.
 Data achieved from laser range sensors are also popular due to their robustness to environmental variations and faster process time
 \cite{pronobis2011phd}.
 Several researchers in computer vision community addressed the problem of place classification, some examples are 
 reviewed here.
 
 Li and Perona in \cite{Li:2005:BHM:1068508.1069129} represented images of scenes as a collection of local regions 
 called codewords achieved from unsupervised learning. 
 Similarly, Lazebnik et al. in \cite{1641019} extended a bag-of-words approach in a computationally efficient way
 by introducing a spacial pyramid which contains approximate global geometric correspondences between local features.
  
 Olivia and Torralba proposed a scene representation called gist of the scene in \cite{oliva2006building} which is 
 used by Torralba et al. in \cite{TorralbaContextualPriming} and \cite{TrollbaContexBased} for place categorization.
 They used 2D object context information as mentioned in previous Chapters in their work.
 Later Quattoni and Torralba in \cite{quattoni2009recognizing} combined the gist of the scene with local features and 
 reported significant improvement in their results.
 
 Also in robotics, researchers worked through capturing some semantics mostly using laser range data.
 Buschka and Saffiotti classified different parts of grid maps into two categories: rooms and corridors in \cite{buschka2002virtual}. 
 
 There are approaches that mostly rely on object information for place categorization.
 Usually a more fine grained description of a place based on its functionality needs association of some key objects 
 for each place category.
 As mentioned in Chapter \ref{Introduction.ch}, rooms are usually classified into categories like $Office$ or 
 $kitchen$ based on their functionality which is tightly dependent to the objects found within them.
 
In \cite{Vasudevan2007359}, Vasudevan et al. suggest a hierarchical probabilistic representation of space using object
information. 
They use SIFT features to detect and recognize objects then create a local probabilistic object graph for the place. 
They also detect doorways to separate rooms from each other in their map. 
Using object graphs for each visited place they make a global topological representation of an environment.

\cite{ranganathan2007semantic} and \cite{P.Viswanathan} also used object based approaches using models trained in 
advance in a supervised manner. 
In \cite{ranganathan2007semantic}, Ranganathan and Dellaert train object models using visual features capturing their 
shape and appearance from roughly segmented and labeled images. 
They also added 3D locations of the objects using stereo range data.
 
Pronobis in \cite{pronobis2011phd} integrated multiple visual cues with geometric information from laser range data. 
Object detection in association of a common-sense knowledge database completes the information needed to determine the category of places in his semantic mapping system.
 
\subsection{Ground Truth}
\label{GroundTruth.sec}
As mentioned before, a ground truth is needed to make the relation between the class of a place and object contexts 
present in it.
This knowledge can be achieved by analyzing available common-sense knowledge databases like \bf{Open Mind} \cite{OpenMind}, 
popular search engines such as Google Image Search, image repositories like Flicker or directly from a human user 
inputs. 

Viswanathan et al. in \cite{P.Viswanathan}, have performed an automated learning of object-place relations on an on-line annotated database such as $Label$ $me$. 
Then object detectors are trained on some of the most frequently occurring objects.
In our case instead of objects we train their context detectors.

Based on the dependencies between place category and its objects also between different object categories we can make 
probabilistic models that represent these relations and dependencies.
Some objects are more discriminative for a place category which should have a more significant role in deciding the 
category of the place.
For instance, a water tap in the kitchen or its sink as its context are quite discriminative for this place category.

\begin{equation}
 \label{sDepend.eq}
 w_i = p(R_c \mid O_i) 
\end{equation}

Equation \ref{sDepend.eq} shows the dependency between a room category ($R_c$) and each object category ($O_i$).
Sometime, the dependency lies on a combination of objects.
A combination of shelves and books can be discriminative for an office, while shelves and dishes are good for kitchen.

\begin{equation}
 \label{mDepend.eq}
 w_{i^{'}} = p(R_c \mid O_j,O_k)
\end{equation}

Dependency of the room category to a combination of object categories is shown in equation \ref{mDepend.eq}. 
A model can be achieved like the one depicted in figure ~\ref{RoomObjectDepend.figure} in which singular dependencies are shown by continuous 
lines and combinational dependency is illustrated by dashed line.
A normalized weight can be computed based on this analysis for each object category.  

 \begin{figure}[t]
  \centerpsw{RoomObjectDepend}{0.65\columnwidth}
  \caption[Probabilistic graphical model of place-object.]
  {This figure shows dependencies between a room category and key object classes, dashed lines shows dependency to a combination of object categories.}
  \label{RoomObjectDepend.figure}
\end{figure}

 
\subsection{Benefits of Object Context for Place Classification}
\label{BenefitsofObjectContextforPlaceClassification.sec}
Some of the benefits of employing $context$ $models$ in place classification can be briefly mentioned as follows: 
\begin{itemize}
 \item It provides a simple way to include human annotations from common-sense knowledge bases into place 
 representation.
 %\item It is a way of including spatial relations into place representation.
 \item It brings in object information while no fine grained object detection is needed.
 \item It is a way to make place models more universal representing them based on object context information instead
 of object information which gives a generalization to the model.
 
\end{itemize}


\subsection{Place classification}
\label{Place classification.sec}
A method is proposed here to compute a score for each place category based on the amount of objects possible to be
found in a place. 
In each place category based on its ground truth learned from a lot of examples we compute a weighted sum of the 
scores of each key object class.
The score for place categories should represent the followings:

\begin{itemize}
 \item Probability of presence of an object class in a place.
 \item Weight or ratio for this presence.
 \item Based on a ground truth a probability of being an specific place category.
\end{itemize}

Applying context models on the point-cloud of a room, we get a subset of points with highest probability of being 
the context. 
In other words, we can estimate the amount of points belonging to a context category in that point-cloud.
An average probability for context points belonging to each context category can be computed and assigned to that category.
Using the ratio of points in each category and the value of the average probability a score is estimated for each context
category (Equation \ref{contextScore.eq}).

\begin{equation}
 \label{contextScore.eq}
Sc_i = Avg(p(c_i)) * \frac{points \in c_i}{points} 
\end{equation}

$Sc_i$ is the score for object category $i$, $Avg(x)$ computes average and the last element of the equation shows the ratio
of points in context category $i$ with respect to all points in the point-cloud. 
Employing weights resulted from ground truth analysis (\ref{GroundTruth.sec}) a score for each place category can be 
estimated.

\begin{equation}
 \label{PlaceScore.eq}
 Spc_j = w_1 * Sc_1 + w_2 * Sc_2 \ldots + w_k * Sc_k
\end{equation}

In equation \ref{PlaceScore.eq}, $Spc_j$ is the score for place category $j$, $k$ is the number of key object categories.
The corresponding weight for each object category is shown by $w_k$. 
Both scores should be normalized to a range between zero and one to be comparable for different place categories. 
Here we used object category and context category as equivalents.
It is noticeable that the proposed equations meant to introduce the general relation between elements.
The exact relations need to be achieved through experiments. 





\section{Future work}

There are some suggestions for improvements and some potential future works which are mentioned here.
%\subsection*{improvements}
\begin{itemize}
 \item The quality of the point-clouds are an important factor, so using better means and methods to capture and build 
 point-clouds with higher quality makes an improvement in results. Doing some pre-process like smoothing and removing noise
 is even a more practical improvement.
 \item Larger train set with more variety of places and object instances is essential. Gathering data in a smart way
  for the train set which considers several possible situations can be more efficient than just increasing the number of samples.
 \item Some more features can be added to make the model more descriptive and discriminative. 
  Visual features should be considered for this improvement.
 \item A comparative evaluation can be done between object detection by a known high performance detector with using context model
  and without.
 \item Develop the system to be able to run in real time and on-line situations.
 \item Employ and experiment place categorization using this model and the proposed method.
\end{itemize} 
